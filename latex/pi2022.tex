\documentclass[aspectratio=169]{beamer}
\usepackage{polski}
\usepackage[utf8]{inputenc}
\usepackage{amssymb}
\usepackage{amsmath}
\usepackage{amsthm}
\usepackage{xcolor}
\usefonttheme[onlymath]{serif}
\usepackage[ddmmyyyy,hhmmss]{datetime}
\renewcommand{\dateseparator}{.}
\usepackage{animate}
\usepackage{fontawesome}
\usepackage{import}
\usepackage{animate}
\usepackage[noadjust]{cite}
\usepackage{hyperref}
\usepackage{mathtools}

\newcommand{\enumsymbol}{$\triangleright$}
\newcommand{\enumsymbolsec}{$\circ$}

\newcommand{\arctg}{\mathrm{arctg}}

\newcommand{\myEmail}{vil02@o2.pl}
\newcommand{\myLinkedinLink}{https://www.linkedin.com/in/piotr-idzik-34b572151/}
\newcommand{\myGithubLink}{https://github.com/vil02/}

\newcommand{\myLinkedin}{\href{\myLinkedinLink}{\faLinkedinSquare}}
\newcommand{\myGithub}{\href{\myGithubLink}{\faGithubSquare}}

\newcommand{\importFromTmpData}[1]{\subimport{\tmpDataFolder}{#1}}

\renewcommand{\GenericWarning}[2]{\GenericError{#1}{#2}{}{This warning has been turned into a fatal error.}}

\let\oldsqrt\sqrt{}
\def\sqrt{\mathpalette\DHLhksqrt}
\def\DHLhksqrt#1#2{%
\setbox0=\hbox{$#1\oldsqrt{#2\,}$}\dimen0=\ht0
\advance\dimen0-0.2\ht0 % chktex 8
\setbox2=\hbox{\vrule height\ht0 depth -\dimen0}%
{\box0\lower0.4pt\box2}}

\newcommand{\paren}[1]{\!\left(#1 \right)}
\newcommand{\email}[1]{\href{mailto: #1}{\texttt{#1}}}
\newcommand{\link}[1]{\href{#1}{\texttt{#1}}}
\renewcommand*\footnoterule{}

\newcommand{\tmpDataFolder}{../tmp_data/} % folder path for all python generated data; relative to latex main file
\newcommand{\curveRepresentationExampleTex}{curve_representation_example.tex}
\newcommand{\convexExamplesTex}{convex_sets_examples.tex}
\newcommand{\lengthInsideExampleTex}{length_inside_example.tex}
\newcommand{\lengthInsideExampleNonconvexTex}{length_inside_example_nonconvex.tex}
\newcommand{\escapeFromCircleTex}{escape_from_circle.tex}

\newcommand{\escapeFromCircleLogoTex}{escape_from_circle_logo.tex}
\newcommand{\escapeFromCircleAzimuthTex}{escape_from_circle_azimuth.tex}
\newcommand{\escapeFromCircleConvPlotTex}{escape_from_circle_conv_plot.tex}

\newcommand{\escapeFromCircleLogoFixedTex}{escape_from_circle_logo_fixed.tex}
\newcommand{\escapeFromCircleAzimuthFixedTex}{escape_from_circle_azimuth_fixed.tex}
\newcommand{\escapeFromCircleConvPlotFixedTex}{escape_from_circle_conv_plot_fixed.tex}

\newcommand{\escapeFromRectangleLogoTex}{escape_from_rectangle_logo.tex}
\newcommand{\escapeFromRectangleAzimuthTex}{escape_from_rectangle_azimuth.tex}
\newcommand{\escapeFromRectangleConvPlotTex}{escape_from_rectangle_conv_plot.tex}

\newcommand{\escapeFromFreeSquareTex}{escape_from_free_square.tex}
\newcommand{\crystalsExamplesTex}{crystals_example.tex}

\newcommand{\tmpColor}{red}


\usetheme{Montpellier}
\author{\texorpdfstring{\href{\myLinkedinLink}{Piotr Idzik} \\ \email{\myEmail}}{Piotr Idzik}}
\title{Problemy optymalizacyjne i symulowane wyżarzanie}
\date{Karlsruhe, Katowice, 14.03.2022}
\setbeamerfont{page number in head/foot}{size=\large, shape=\ttfamily\color{black}}

\setbeamertemplate{navigation symbols}
{%
  \vbox{%
  \hbox{\insertbackfindforwardnavigationsymbol}}%
}

\expandafter\def\expandafter\insertshorttitle\expandafter{%
  \insertshorttitle\hfill%
  \texttt{\insertframenumber}}

\begin{document}
\begin{frame}[plain]
\maketitle

\myLinkedin{}
\myGithub{}
\hfill \textcolor[rgb]{0.85,0.85,0.85}{\texttt{\tiny{ver. \today\ \currenttime}}}
\end{frame}
\section{Problemy optymalizacyjne}
\begin{frame}
Jaki prostokąt o obwodzie 2 ma największe pole powierzchni?
\pause{}
Jaka jest jego wartość?
\pause{}
\begin{equation} \label{eq::formula_for_b}
2a+2b = 2 \pause \quad \Rightarrow \quad a+b = 1 \pause \quad \Rightarrow \quad b=1-a
\end{equation}
\pause{}
\begin{equation*}
P = ab \pause \stackrel{\eqref{eq::formula_for_b}}{=} a\paren{1-a}
\end{equation*}
\pause{}
\begin{equation*}
\max_{a \in [0, 1]} a\paren{1-a}
\end{equation*}
\pause{}
\begin{figure}
  \center{}
  \importFromTmpData{\rectangleAreaPlotTex}
\end{figure}
\end{frame}

\subsection{Optima lokalne i globalne}
\importFromTmpData{\localOptimumsTex}

\section{Wyżarzanie}
\subsection{Struktura krystaliczna metali}
\importFromTmpData{\crystalsExamplesTex}

\section{Symulowane wyżarzanie}
\begin{frame}
\[\min_{s \in S}F\paren{s}\]
\begin{enumerate}
\item<7-> $T \coloneqq T_{\max}$, $s \coloneqq s_0$
\item<2-> $s^\prime \coloneqq \texttt{NowyStan}\paren{s, T}$\label{item::new_state}
\item<3-> Jeżeli $F\paren{s^\prime} < F\paren{s}$ to \
\[s \coloneqq s^\prime\]
\uncover<4->{w przeciwnym wypadku
\[s \coloneqq s^\prime, \text{ z prawdopodobieństwem } P\paren{F(s), F(s^\prime), T}\]}
\item<5-> $T \coloneqq \texttt{NowaTemperatura}\paren{T}$
\item<6-> Idź do~\ref{item::new_state} o ile niespełnione kryterium stopu.
\end{enumerate}
\uncover<8->{$\texttt{NowaTemperatura}\paren{T} = \alpha T$, gdzie $\alpha \in (0, 1)$,}
\uncover<9->{$P\paren{F(s), F(s^\prime), T} = e^{\frac{F\paren{s^\prime}-F\paren{s}}{kT}}$, gdzie $k>0$.}
\end{frame}

\section{Ucieczka ze zbiorów wypukłych}

\begin{frame}
  \begin{figure}
    \center{}
    \importFromTmpData{\lostInForestExTex}
  \end{figure}
  \pause{}
  Richard E. Bellman, 1956,~\cite{Bellman1956} (zob.~\cite{Williams2000, FinchWetzel2004})
\end{frame}

\subsection{Zbiory wypukłe}
\importFromTmpData{\convexExamplesTex}

\subsection{Reprezentacja krzywej}
\importFromTmpData{\curveRepresentationExampleTex}


\begin{frame}
\begin{columns}
  \begin{column}{0.5\textwidth}
    \begin{figure}
      \importFromTmpData{\curveComparisonLogoTex}
    \end{figure}
  \end{column}
  \begin{column}{0.5\textwidth}
    \begin{figure}
      \importFromTmpData{\curveComparisonAzimuthTex}
    \end{figure}
  \end{column}
\end{columns}
\end{frame}

\subsection{Jak mierzyć długość krzywej wewnątrz zbioru?}
\importFromTmpData{\lengthInsideExampleTex}
\subsection{Wypukłość jest istotna}
\importFromTmpData{\lengthInsideExampleNonconvexTex}

\section{Ucieczka z koła}

\begin{frame}
\begin{figure}
  \importFromTmpData{\escapeFromCircleExTex}
\end{figure}
\end{frame}

\begin{frame}
\begin{columns}
\begin{column}{0.5\textwidth}
  \begin{figure}
    \importFromTmpData{\escapeFromCircleLogoTex}
  \end{figure}
\end{column}
\begin{column}{0.5\textwidth}
  \begin{figure}
    \importFromTmpData{\escapeFromCircleAzimuthTex}
  \end{figure}
\end{column}
\end{columns}
\end{frame}

\begin{frame}
\begin{figure}
  \importFromTmpData{\escapeFromCircleConvPlotTex}
\end{figure}
\end{frame}

\begin{frame}
\begin{columns}
\begin{column}{0.5\textwidth}
  \begin{figure}
    \importFromTmpData{\escapeFromCircleLogoFixedTex}
  \end{figure}
\end{column}
\begin{column}{0.5\textwidth}
  \begin{figure}
    \importFromTmpData{\escapeFromCircleAzimuthFixedTex}
  \end{figure}
\end{column}
\end{columns}
\end{frame}

\begin{frame}
\begin{figure}
  \importFromTmpData{\escapeFromCircleConvPlotFixedTex}
\end{figure}
\end{frame}

\section{Ucieczka z prostokąta}

\begin{frame}
\begin{figure}
  \importFromTmpData{\escapeFromRectangleExTex}
\end{figure}
\end{frame}

\begin{frame}
\begin{columns}
\begin{column}{0.5\textwidth}
  \begin{figure}
    \importFromTmpData{\escapeFromRectangleLogoTex}
  \end{figure}
\end{column}
\begin{column}{0.5\textwidth}
  \begin{figure}
    \importFromTmpData{\escapeFromRectangleAzimuthTex}
  \end{figure}
\end{column}
\end{columns}
\end{frame}

\begin{frame}
\begin{figure}
  \importFromTmpData{\escapeFromRectangleConvPlotTex}
\end{figure}
\end{frame}

\section{Ucieczka z półpłaszczyzny}

\importFromTmpData{\escapeFromHalfplaneExTex}

\subsection{Funkcja kosztu}

\begin{frame}
\begin{equation*}
\uncover<+->{A_1, A_2, \ldots, A_n}
\end{equation*}
\begin{equation*}
\uncover<+->{l\paren{s, A_1}, l\paren{s, A_2},  \ldots, l\paren{s, A_n}}
\end{equation*}

\begin{align*}
\uncover<+->{F(s) &= \max\paren{l\paren{s, A_1}, l\paren{s, A_2},  \ldots, l\paren{s, A_n}}} \\
\uncover<+->{G(s) &= l\paren{s, A_1} + l\paren{s, A_2} + \cdots + l\paren{s, A_n}}
\end{align*}

\end{frame}

\subsection{Wyniki eksperymentów}

\begin{frame}
  \begin{figure}
    \importFromTmpData{\escapeFromHalfplaneAzimuthTex}
  \end{figure}
\end{frame}

\begin{frame}
  \begin{figure}
    \importFromTmpData{\escapeFromHalfplanePointTex}
  \end{figure}
\end{frame}

\begin{frame}
  \begin{figure}
    \importFromTmpData{\escapeFromHalfplaneConvPlotTex}
  \end{figure}
\end{frame}

\subsection{Rozwiązanie analityczne}

\begin{frame}
\begin{equation*}
\sqrt{3}+\frac{7}{6}\pi+1 \approx 6.3972\ldots
\end{equation*}
\pause{}
John R. Isbell, 1956~\cite{Isbell1957} oraz Henri Joris, 1980~\cite{Joris1980, Finch2019}
\end{frame}

\section{Ucieczka z \textit{paska}}

\importFromTmpData{\escapeFromStripExTex}

\subsection{Wyniki eksperymentów}

\begin{frame}
  \begin{figure}
    \importFromTmpData{\escapeFromStripAzimuthTex}
  \end{figure}
\end{frame}

\begin{frame}
  \begin{figure}
    \importFromTmpData{\escapeFromStripConvPlotTex}
  \end{figure}
\end{frame}

\subsection{Rozwiązanie analityczne}

\begin{frame}
\begin{equation*}
2\paren{\frac{\pi}{2}-\varphi-2\psi+\tg\varphi+\tg\psi} \approx 2.27829\ldots,
\end{equation*}
gdzie
\begin{align*}
  \varphi &= \arcsin\paren{\frac{1}{6}+\frac{4}{3}\sin\paren{\frac{1}{3}\arcsin\frac{17}{64}}}, \\
  \psi &= \arctg\paren{\frac{1}{2}\sec\varphi}.
\end{align*}
\pause{}
Wiktor Zalgaller, 1961, zob.~\cite{Finch2019b}
\end{frame}

\section{Podsumowanie}
\begin{frame}
Algorytm symulowanego wyżarzanie jest
\begin{itemize}
\item<+-> \textit{jedynie} algorytmem heurystycznym,
\item<+-> prosty w implementacji i użyciu,
\item<+-> powszechnie stosowany.
\end{itemize}
\uncover<+->{Reprezentacja danych jest istotna.}

\uncover<+->{\vspace*{2cm} Prezentacja: \link{\myGithubLink pi2022/}}
\end{frame}

\section{Bibliografia}
\begin{frame}[allowframebreaks]
  \bibliography{bib_data}{}
  \bibliographystyle{plain}
\end{frame}

\end{document}
